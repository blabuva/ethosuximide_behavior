\documentclass[]{article}
\usepackage{lmodern}
\usepackage{amssymb,amsmath}
\usepackage{ifxetex,ifluatex}
\usepackage{fixltx2e} % provides \textsubscript
\ifnum 0\ifxetex 1\fi\ifluatex 1\fi=0 % if pdftex
  \usepackage[T1]{fontenc}
  \usepackage[utf8]{inputenc}
\else % if luatex or xelatex
  \ifxetex
    \usepackage{mathspec}
  \else
    \usepackage{fontspec}
  \fi
  \defaultfontfeatures{Ligatures=TeX,Scale=MatchLowercase}
\fi
% use upquote if available, for straight quotes in verbatim environments
\IfFileExists{upquote.sty}{\usepackage{upquote}}{}
% use microtype if available
\IfFileExists{microtype.sty}{%
\usepackage{microtype}
\UseMicrotypeSet[protrusion]{basicmath} % disable protrusion for tt fonts
}{}
\usepackage[margin=1in]{geometry}
\usepackage{hyperref}
\hypersetup{unicode=true,
            pdftitle={Test Novel Object Analysis},
            pdfauthor={Alexa \& Viktoria},
            pdfborder={0 0 0},
            breaklinks=true}
\urlstyle{same}  % don't use monospace font for urls
\usepackage{color}
\usepackage{fancyvrb}
\newcommand{\VerbBar}{|}
\newcommand{\VERB}{\Verb[commandchars=\\\{\}]}
\DefineVerbatimEnvironment{Highlighting}{Verbatim}{commandchars=\\\{\}}
% Add ',fontsize=\small' for more characters per line
\usepackage{framed}
\definecolor{shadecolor}{RGB}{248,248,248}
\newenvironment{Shaded}{\begin{snugshade}}{\end{snugshade}}
\newcommand{\KeywordTok}[1]{\textcolor[rgb]{0.13,0.29,0.53}{\textbf{#1}}}
\newcommand{\DataTypeTok}[1]{\textcolor[rgb]{0.13,0.29,0.53}{#1}}
\newcommand{\DecValTok}[1]{\textcolor[rgb]{0.00,0.00,0.81}{#1}}
\newcommand{\BaseNTok}[1]{\textcolor[rgb]{0.00,0.00,0.81}{#1}}
\newcommand{\FloatTok}[1]{\textcolor[rgb]{0.00,0.00,0.81}{#1}}
\newcommand{\ConstantTok}[1]{\textcolor[rgb]{0.00,0.00,0.00}{#1}}
\newcommand{\CharTok}[1]{\textcolor[rgb]{0.31,0.60,0.02}{#1}}
\newcommand{\SpecialCharTok}[1]{\textcolor[rgb]{0.00,0.00,0.00}{#1}}
\newcommand{\StringTok}[1]{\textcolor[rgb]{0.31,0.60,0.02}{#1}}
\newcommand{\VerbatimStringTok}[1]{\textcolor[rgb]{0.31,0.60,0.02}{#1}}
\newcommand{\SpecialStringTok}[1]{\textcolor[rgb]{0.31,0.60,0.02}{#1}}
\newcommand{\ImportTok}[1]{#1}
\newcommand{\CommentTok}[1]{\textcolor[rgb]{0.56,0.35,0.01}{\textit{#1}}}
\newcommand{\DocumentationTok}[1]{\textcolor[rgb]{0.56,0.35,0.01}{\textbf{\textit{#1}}}}
\newcommand{\AnnotationTok}[1]{\textcolor[rgb]{0.56,0.35,0.01}{\textbf{\textit{#1}}}}
\newcommand{\CommentVarTok}[1]{\textcolor[rgb]{0.56,0.35,0.01}{\textbf{\textit{#1}}}}
\newcommand{\OtherTok}[1]{\textcolor[rgb]{0.56,0.35,0.01}{#1}}
\newcommand{\FunctionTok}[1]{\textcolor[rgb]{0.00,0.00,0.00}{#1}}
\newcommand{\VariableTok}[1]{\textcolor[rgb]{0.00,0.00,0.00}{#1}}
\newcommand{\ControlFlowTok}[1]{\textcolor[rgb]{0.13,0.29,0.53}{\textbf{#1}}}
\newcommand{\OperatorTok}[1]{\textcolor[rgb]{0.81,0.36,0.00}{\textbf{#1}}}
\newcommand{\BuiltInTok}[1]{#1}
\newcommand{\ExtensionTok}[1]{#1}
\newcommand{\PreprocessorTok}[1]{\textcolor[rgb]{0.56,0.35,0.01}{\textit{#1}}}
\newcommand{\AttributeTok}[1]{\textcolor[rgb]{0.77,0.63,0.00}{#1}}
\newcommand{\RegionMarkerTok}[1]{#1}
\newcommand{\InformationTok}[1]{\textcolor[rgb]{0.56,0.35,0.01}{\textbf{\textit{#1}}}}
\newcommand{\WarningTok}[1]{\textcolor[rgb]{0.56,0.35,0.01}{\textbf{\textit{#1}}}}
\newcommand{\AlertTok}[1]{\textcolor[rgb]{0.94,0.16,0.16}{#1}}
\newcommand{\ErrorTok}[1]{\textcolor[rgb]{0.64,0.00,0.00}{\textbf{#1}}}
\newcommand{\NormalTok}[1]{#1}
\usepackage{graphicx,grffile}
\makeatletter
\def\maxwidth{\ifdim\Gin@nat@width>\linewidth\linewidth\else\Gin@nat@width\fi}
\def\maxheight{\ifdim\Gin@nat@height>\textheight\textheight\else\Gin@nat@height\fi}
\makeatother
% Scale images if necessary, so that they will not overflow the page
% margins by default, and it is still possible to overwrite the defaults
% using explicit options in \includegraphics[width, height, ...]{}
\setkeys{Gin}{width=\maxwidth,height=\maxheight,keepaspectratio}
\IfFileExists{parskip.sty}{%
\usepackage{parskip}
}{% else
\setlength{\parindent}{0pt}
\setlength{\parskip}{6pt plus 2pt minus 1pt}
}
\setlength{\emergencystretch}{3em}  % prevent overfull lines
\providecommand{\tightlist}{%
  \setlength{\itemsep}{0pt}\setlength{\parskip}{0pt}}
\setcounter{secnumdepth}{0}
% Redefines (sub)paragraphs to behave more like sections
\ifx\paragraph\undefined\else
\let\oldparagraph\paragraph
\renewcommand{\paragraph}[1]{\oldparagraph{#1}\mbox{}}
\fi
\ifx\subparagraph\undefined\else
\let\oldsubparagraph\subparagraph
\renewcommand{\subparagraph}[1]{\oldsubparagraph{#1}\mbox{}}
\fi

%%% Use protect on footnotes to avoid problems with footnotes in titles
\let\rmarkdownfootnote\footnote%
\def\footnote{\protect\rmarkdownfootnote}

%%% Change title format to be more compact
\usepackage{titling}

% Create subtitle command for use in maketitle
\newcommand{\subtitle}[1]{
  \posttitle{
    \begin{center}\large#1\end{center}
    }
}

\setlength{\droptitle}{-2em}

  \title{Test Novel Object Analysis}
    \pretitle{\vspace{\droptitle}\centering\huge}
  \posttitle{\par}
    \author{Alexa \& Viktoria}
    \preauthor{\centering\large\emph}
  \postauthor{\par}
      \predate{\centering\large\emph}
  \postdate{\par}
    \date{December 8th, 2018}


\begin{document}
\maketitle

Load libraries

\begin{Shaded}
\begin{Highlighting}[]
\CommentTok{# tidyverse includes the packages readr, dplyr, ggplot2 ... etc.}
\KeywordTok{library}\NormalTok{(tidyverse)}

\CommentTok{# ggbeeswarm includes geom_beeswarm()}
\KeywordTok{library}\NormalTok{(ggbeeswarm)}
\end{Highlighting}
\end{Shaded}

\subsection{Read in the raw data}\label{read-in-the-raw-data}

Read in the novel object test raw data
(\href{Analysis\%20Output-VA\%20Novel\%20Object.csv}{\textbf{Analysis
Output-VA Novel Object.csv}}):

\begin{Shaded}
\begin{Highlighting}[]
\CommentTok{# Import novel object data}
\NormalTok{novelObjectRaw <-}\StringTok{ }\KeywordTok{read_csv}\NormalTok{(}\StringTok{"Analysis Output-VA Novel Object.csv"}\NormalTok{, }\DataTypeTok{skip =} \DecValTok{4}\NormalTok{, }
                            \DataTypeTok{col_names =}\NormalTok{ header)}
\CommentTok{# Look at Novel Object data}
\NormalTok{novelObjectRaw}
\end{Highlighting}
\end{Shaded}

\begin{verbatim}
## # A tibble: 40 x 13
##    `Object Type` `Trial # NO` `Trial # OF` `Novel Object` `Familiar Objec~
##    <chr>                <dbl>        <dbl> <chr>          <chr>           
##  1 Familiar                 1            1 object 1       object 2        
##  2 Familiar                 2            2 object 1       object 2        
##  3 Familiar                 3            3 object 1       object 2        
##  4 Familiar                 4            4 object 1       object 2        
##  5 Novel                    5            1 object 1       object 2        
##  6 Novel                    6            2 object 1       object 2        
##  7 Novel                    7            3 object 1       object 2        
##  8 Novel                    8            4 object 1       object 2        
##  9 Familiar                 9            5 object 1       object 2        
## 10 Familiar                10            6 object 1       object 2        
## # ... with 30 more rows, and 8 more variables: `Distance to objects:object
## #   1/Nose-point:Mean` <dbl>, `Distance to objects:object
## #   2/Nose-point:Mean` <dbl>, `Nose touching objects:object
## #   1/Nose-point:Duration` <dbl>, `Nose touching objects:object
## #   1/Nose-point:Frequency` <dbl>, `Nose touching objects:object
## #   1/Nose-point:Latency to first` <dbl>, `Nose touching objects:object
## #   2/Nose-point:Duration` <dbl>, `Nose touching objects:object
## #   2/Nose-point:Frequency` <dbl>, `Nose touching objects:object
## #   2/Nose-point:Latency to first` <dbl>
\end{verbatim}

Read in the data containing the treatment and sex information for each
animal (\textbf{Injections Data.csv}):

\begin{Shaded}
\begin{Highlighting}[]
\CommentTok{# Import injections data}
\NormalTok{injectionsDataRaw <-}\StringTok{ }\KeywordTok{read_csv}\NormalTok{(}\StringTok{"Injections Data.csv"}\NormalTok{, }\DataTypeTok{n_max =} \DecValTok{20}\NormalTok{)}

\CommentTok{# Take a look at the injections data}
\NormalTok{injectionsDataRaw}
\end{Highlighting}
\end{Shaded}

\begin{verbatim}
## # A tibble: 20 x 29
##    Treatment Sex   Tag   `12-Oct` `13-Oct` `14-Oct` `15-Oct` `16-Oct`
##        <dbl> <chr> <chr>    <dbl>    <dbl>    <dbl>    <dbl>    <dbl>
##  1         0 F     1L          25       27       31       34       36
##  2         1 M     2L          27       29       33       37       36
##  3         0 M     3L          25       27       31       33       34
##  4         1 F     1R          23       24       28       30       33
##  5         0 M     2R          25       27       31       34       35
##  6         1 F     3R          21       23       25       27       30
##  7         0 M     1L-1R       27       30       34       36       37
##  8         1 M     1L-2R       28       29       32       36       38
##  9         0 F     2L-2R       26       27       31       33       35
## 10         1 F     2L-1R       27       28       31       34       36
## 11         0 F     1L          21       25       27       31       32
## 12         1 F     2L          23       25       28       30       33
## 13         0 M     3L          23       27       30       32       36
## 14         1 M     1R          24       26       28       32       34
## 15         0 M     2L-1R       23       26       28       31       33
## 16         1 M     3R          24       27       30       33       36
## 17         0 M     1L-1R       25       27       30       33       35
## 18         1 F     1L-2R       22       25       27       30       32
## 19         0 F     2L-2R       23       26       29       31       35
## 20         1 F     2R          23       26       29       30       33
## # ... with 21 more variables: `17-Oct` <dbl>, `18-Oct` <dbl>,
## #   `19-Oct` <dbl>, `20-Oct` <dbl>, `21-Oct` <dbl>, `22-Oct` <dbl>,
## #   `23-Oct` <dbl>, `24-Oct` <dbl>, `25-Oct` <dbl>, `26-Oct` <dbl>,
## #   `27-Oct` <dbl>, `28-Oct` <dbl>, `29-Oct` <dbl>, `30-Oct` <dbl>,
## #   `31-Oct` <dbl>, `Parent Cage #` <dbl>, `Current Cage #` <dbl>, `Trial
## #   # Open Field` <dbl>, `Trial # Familiar Object` <dbl>, `Trial # Novel
## #   Object` <dbl>, Comments <chr>
\end{verbatim}

Read in a table that maps treatment numbers to treatment strings

\begin{Shaded}
\begin{Highlighting}[]
\CommentTok{# Create a dictionary for Treatment}
\NormalTok{treatmentTypeTable <-}\StringTok{ }\KeywordTok{read_csv}\NormalTok{(}\StringTok{"Treatment Dictionary.csv"}\NormalTok{)}

\CommentTok{# Display the table}
\NormalTok{treatmentTypeTable}
\end{Highlighting}
\end{Shaded}

\begin{verbatim}
## # A tibble: 2 x 2
##   Treatment TreatmentStr
##       <dbl> <chr>       
## 1         0 Control     
## 2         1 Ethosuximide
\end{verbatim}

\subsection{Combine and rearrange the
data}\label{combine-and-rearrange-the-data}

Starting from the open field raw data, do the following: 1. Add the
treatment and sex information by matching up the trial number 2. Add the
treatment string be matching up the treatment number 3. Choose the
column corresponding to duration in center and rename it 4. Choose the
column corresponding to duration in border and rename it 5. Make
``Sex''" a categorical variable 6. Make ``Treatment'' a categorical
variable 7. Make ``DurationObjectRatio'' variable

\begin{Shaded}
\begin{Highlighting}[]
\CommentTok{# Rearrange the data}
\NormalTok{novelObjectExpanded <-}\StringTok{ }
\StringTok{    }\NormalTok{novelObjectRaw }\OperatorTok\StringTok{ }
\StringTok{    }\KeywordTok{full_join}\NormalTok{(injectionsDataRaw, }\DataTypeTok{by =} \KeywordTok{c}\NormalTok{(}\StringTok{"Trial # OF"}\NormalTok{ =}\StringTok{ "Trial # Open Field"}\NormalTok{)) }\OperatorTok\StringTok{ }
\StringTok{    }\KeywordTok{full_join}\NormalTok{(treatmentTypeTable, }\DataTypeTok{by =} \StringTok{"Treatment"}\NormalTok{) }\OperatorTok\StringTok{ }
\StringTok{    }\KeywordTok{mutate}\NormalTok{(}\DataTypeTok{DurationObject1 =} \StringTok{`}\DataTypeTok{Nose touching objects:object 1/Nose-point:Duration}\StringTok{`}\NormalTok{) }\OperatorTok\StringTok{ }
\StringTok{    }\KeywordTok{mutate}\NormalTok{(}\DataTypeTok{DurationObject2 =} \StringTok{`}\DataTypeTok{Nose touching objects:object 2/Nose-point:Duration}\StringTok{`}\NormalTok{) }\OperatorTok
\StringTok{    }\KeywordTok{mutate}\NormalTok{(}\DataTypeTok{Sex =} \KeywordTok{factor}\NormalTok{(Sex)) }\OperatorTok
\StringTok{    }\KeywordTok{mutate}\NormalTok{(}\DataTypeTok{TreatmentStr =} \KeywordTok{factor}\NormalTok{(TreatmentStr)) }\OperatorTok\StringTok{ }
\StringTok{    }\KeywordTok{mutate}\NormalTok{(}\DataTypeTok{DurationObjectRatio =}\NormalTok{ DurationObject1 }\OperatorTok{/}\StringTok{ }\NormalTok{DurationObject2) }\OperatorTok\StringTok{ }
\StringTok{    }\KeywordTok{mutate}\NormalTok{(}\DataTypeTok{TrialType =} \KeywordTok{factor}\NormalTok{(}\StringTok{`}\DataTypeTok{Object Type}\StringTok{`}\NormalTok{)) }\OperatorTok\StringTok{ }
\StringTok{    }\KeywordTok{mutate}\NormalTok{(}\DataTypeTok{TrialTypeTreatment =} \KeywordTok{interaction}\NormalTok{(TrialType, TreatmentStr, }\DataTypeTok{sep =} \StringTok{"-"}\NormalTok{))}

\CommentTok{# Take a glimpse of all the variables in the data }
\KeywordTok{glimpse}\NormalTok{(novelObjectExpanded)}
\end{Highlighting}
\end{Shaded}

\begin{verbatim}
## Observations: 40
## Variables: 47
## $ `Object Type`                                                <chr> "...
## $ `Trial # NO`                                                 <dbl> 1...
## $ `Trial # OF`                                                 <dbl> 1...
## $ `Novel Object`                                               <chr> "...
## $ `Familiar Object`                                            <chr> "...
## $ `Distance to objects:object 1/Nose-point:Mean`               <dbl> 2...
## $ `Distance to objects:object 2/Nose-point:Mean`               <dbl> 2...
## $ `Nose touching objects:object 1/Nose-point:Duration`         <dbl> 1...
## $ `Nose touching objects:object 1/Nose-point:Frequency`        <dbl> 2...
## $ `Nose touching objects:object 1/Nose-point:Latency to first` <dbl> 6...
## $ `Nose touching objects:object 2/Nose-point:Duration`         <dbl> 2...
## $ `Nose touching objects:object 2/Nose-point:Frequency`        <dbl> 3...
## $ `Nose touching objects:object 2/Nose-point:Latency to first` <dbl> 1...
## $ Treatment                                                    <dbl> 0...
## $ Sex                                                          <fct> F...
## $ Tag                                                          <chr> "...
## $ `12-Oct`                                                     <dbl> 2...
## $ `13-Oct`                                                     <dbl> 2...
## $ `14-Oct`                                                     <dbl> 3...
## $ `15-Oct`                                                     <dbl> 3...
## $ `16-Oct`                                                     <dbl> 3...
## $ `17-Oct`                                                     <dbl> 3...
## $ `18-Oct`                                                     <dbl> 4...
## $ `19-Oct`                                                     <dbl> 4...
## $ `20-Oct`                                                     <dbl> 4...
## $ `21-Oct`                                                     <dbl> 5...
## $ `22-Oct`                                                     <dbl> 5...
## $ `23-Oct`                                                     <dbl> 6...
## $ `24-Oct`                                                     <dbl> 6...
## $ `25-Oct`                                                     <dbl> 7...
## $ `26-Oct`                                                     <dbl> 8...
## $ `27-Oct`                                                     <dbl> 8...
## $ `28-Oct`                                                     <dbl> 9...
## $ `29-Oct`                                                     <dbl> 9...
## $ `30-Oct`                                                     <dbl> 1...
## $ `31-Oct`                                                     <dbl> 1...
## $ `Parent Cage #`                                              <dbl> 1...
## $ `Current Cage #`                                             <dbl> 1...
## $ `Trial # Familiar Object`                                    <dbl> 1...
## $ `Trial # Novel Object`                                       <dbl> 5...
## $ Comments                                                     <chr> N...
## $ TreatmentStr                                                 <fct> C...
## $ DurationObject1                                              <dbl> 1...
## $ DurationObject2                                              <dbl> 2...
## $ DurationObjectRatio                                          <dbl> 0...
## $ TrialType                                                    <fct> F...
## $ TrialTypeTreatment                                           <fct> F...
\end{verbatim}

Remove the first animal because it was used as a test animal

\begin{Shaded}
\begin{Highlighting}[]
\CommentTok{# Select all but the first row}
\NormalTok{novelObjectFiltered <-}\StringTok{ }
\StringTok{    }\NormalTok{novelObjectExpanded }\OperatorTok\StringTok{ }
\StringTok{    }\KeywordTok{slice}\NormalTok{(}\DecValTok{2}\OperatorTok{:}\KeywordTok{n}\NormalTok{())}

\CommentTok{# Take a look at the data with rows filtered}
\NormalTok{novelObjectFiltered}
\end{Highlighting}
\end{Shaded}

\begin{verbatim}
## # A tibble: 39 x 47
##    `Object Type` `Trial # NO` `Trial # OF` `Novel Object` `Familiar Objec~
##    <chr>                <dbl>        <dbl> <chr>          <chr>           
##  1 Familiar                 2            2 object 1       object 2        
##  2 Familiar                 3            3 object 1       object 2        
##  3 Familiar                 4            4 object 1       object 2        
##  4 Novel                    5            1 object 1       object 2        
##  5 Novel                    6            2 object 1       object 2        
##  6 Novel                    7            3 object 1       object 2        
##  7 Novel                    8            4 object 1       object 2        
##  8 Familiar                 9            5 object 1       object 2        
##  9 Familiar                10            6 object 1       object 2        
## 10 Familiar                11            7 object 1       object 2        
## # ... with 29 more rows, and 42 more variables: `Distance to
## #   objects:object 1/Nose-point:Mean` <dbl>, `Distance to objects:object
## #   2/Nose-point:Mean` <dbl>, `Nose touching objects:object
## #   1/Nose-point:Duration` <dbl>, `Nose touching objects:object
## #   1/Nose-point:Frequency` <dbl>, `Nose touching objects:object
## #   1/Nose-point:Latency to first` <dbl>, `Nose touching objects:object
## #   2/Nose-point:Duration` <dbl>, `Nose touching objects:object
## #   2/Nose-point:Frequency` <dbl>, `Nose touching objects:object
## #   2/Nose-point:Latency to first` <dbl>, Treatment <dbl>, Sex <fct>,
## #   Tag <chr>, `12-Oct` <dbl>, `13-Oct` <dbl>, `14-Oct` <dbl>,
## #   `15-Oct` <dbl>, `16-Oct` <dbl>, `17-Oct` <dbl>, `18-Oct` <dbl>,
## #   `19-Oct` <dbl>, `20-Oct` <dbl>, `21-Oct` <dbl>, `22-Oct` <dbl>,
## #   `23-Oct` <dbl>, `24-Oct` <dbl>, `25-Oct` <dbl>, `26-Oct` <dbl>,
## #   `27-Oct` <dbl>, `28-Oct` <dbl>, `29-Oct` <dbl>, `30-Oct` <dbl>,
## #   `31-Oct` <dbl>, `Parent Cage #` <dbl>, `Current Cage #` <dbl>, `Trial
## #   # Familiar Object` <dbl>, `Trial # Novel Object` <dbl>,
## #   Comments <chr>, TreatmentStr <fct>, DurationObject1 <dbl>,
## #   DurationObject2 <dbl>, DurationObjectRatio <dbl>, TrialType <fct>,
## #   TrialTypeTreatment <fct>
\end{verbatim}

\begin{Shaded}
\begin{Highlighting}[]
\CommentTok{# Select variables that will be analyzed}
\NormalTok{novelObjectData <-}\StringTok{ }
\StringTok{    }\NormalTok{novelObjectFiltered }\OperatorTok\StringTok{ }
\StringTok{    }\KeywordTok{select}\NormalTok{(TrialType, TreatmentStr, Sex, }
\NormalTok{           DurationObject1, DurationObject2, DurationObjectRatio, Tag, TrialTypeTreatment)}

\CommentTok{# Take a look at the data with columns filtered}
\NormalTok{novelObjectData}
\end{Highlighting}
\end{Shaded}

\begin{verbatim}
## # A tibble: 39 x 8
##    TrialType TreatmentStr Sex   DurationObject1 DurationObject2
##    <fct>     <fct>        <fct>           <dbl>           <dbl>
##  1 Familiar  Control      F                40.2            37.0
##  2 Familiar  Ethosuximide F                28.0            36.2
##  3 Familiar  Control      M                14.6            23.8
##  4 Novel     Control      F                48.0            28.8
##  5 Novel     Control      F                44.8            20.4
##  6 Novel     Ethosuximide F                52.9            25.4
##  7 Novel     Control      M                48.8            24.2
##  8 Familiar  Control      M                40.4            20.2
##  9 Familiar  Ethosuximide F                51.9            33.8
## 10 Familiar  Ethosuximide F                64.9            54.1
## # ... with 29 more rows, and 3 more variables: DurationObjectRatio <dbl>,
## #   Tag <chr>, TrialTypeTreatment <fct>
\end{verbatim}

\subsection{Make plots}\label{make-plots}

\paragraph{Use Duration at Object 1 / Duration at Object 2 (Duration
Object
Ratio)}\label{use-duration-at-object-1-duration-at-object-2-duration-object-ratio}

Plot the Duration Object Ratio versus Trial Type as a box plot, colored
by Treatment

\begin{Shaded}
\begin{Highlighting}[]
\CommentTok{# Add the data to the canvas}
\NormalTok{durationRatioPlot2 <-}\StringTok{ }
\StringTok{    }\KeywordTok{ggplot}\NormalTok{(novelObjectData, }\KeywordTok{aes}\NormalTok{(TrialType, DurationObjectRatio, }\DataTypeTok{color =}\NormalTok{ TreatmentStr)) }\OperatorTok{+}
\StringTok{    }\KeywordTok{xlab}\NormalTok{(}\StringTok{"Trial Type"}\NormalTok{) }\OperatorTok{+}\StringTok{ }\KeywordTok{ylab}\NormalTok{(}\StringTok{"Duration Object Ratio"}\NormalTok{) }\OperatorTok{+}
\StringTok{    }\KeywordTok{ggtitle}\NormalTok{(}\StringTok{"Novel Object Test for Ethosuximide-treated animals"}\NormalTok{)}

\CommentTok{# Plot as a combination of jitter and box plot}
\NormalTok{durationRatioPlot2 }\OperatorTok{+}\StringTok{ }\KeywordTok{geom_boxplot}\NormalTok{(}\DataTypeTok{outlier.color =} \StringTok{'purple'}\NormalTok{)}
\end{Highlighting}
\end{Shaded}

\includegraphics{test_novel_object_analysis_files/figure-latex/unnamed-chunk-9-1.pdf}

\begin{Shaded}
\begin{Highlighting}[]
\CommentTok{# Save the plot}
\KeywordTok{ggsave}\NormalTok{(}\StringTok{"novelObject_durationObjectRatio_boxplot.png"}\NormalTok{)}
\end{Highlighting}
\end{Shaded}

\begin{Shaded}
\begin{Highlighting}[]
\CommentTok{# Add the data to the canvas}
\NormalTok{durationRatioPlot1 <-}\StringTok{ }
\StringTok{    }\KeywordTok{ggplot}\NormalTok{(novelObjectData, }\KeywordTok{aes}\NormalTok{(TrialTypeTreatment, DurationObjectRatio, }\DataTypeTok{color =}\NormalTok{ TreatmentStr)) }\OperatorTok{+}
\StringTok{    }\KeywordTok{xlab}\NormalTok{(}\StringTok{"Trial Type"}\NormalTok{) }\OperatorTok{+}\StringTok{ }\KeywordTok{ylab}\NormalTok{(}\StringTok{"Duration Object Ratio"}\NormalTok{) }\OperatorTok{+}
\StringTok{    }\KeywordTok{ggtitle}\NormalTok{(}\StringTok{"Novel Object Test for Ethosuximide-treated animals"}\NormalTok{)}

\CommentTok{# Plot as a combination of jitter and box plot}
\NormalTok{durationRatioPlot1 }\OperatorTok{+}\StringTok{ }\KeywordTok{geom_boxplot}\NormalTok{(}\DataTypeTok{outlier.color =} \StringTok{'purple'}\NormalTok{) }\OperatorTok{+}
\StringTok{    }\KeywordTok{geom_jitter}\NormalTok{(}\DataTypeTok{alpha =} \FloatTok{0.75}\NormalTok{)}
\end{Highlighting}
\end{Shaded}

\includegraphics{test_novel_object_analysis_files/figure-latex/unnamed-chunk-10-1.pdf}

\begin{Shaded}
\begin{Highlighting}[]
\CommentTok{# Save the plot}
\KeywordTok{ggsave}\NormalTok{(}\StringTok{"novelObject_durationObjectRatio_boxplot_jitter.png"}\NormalTok{)}
\end{Highlighting}
\end{Shaded}

Plot the Duration Object Ratio versus Trial Type as a violin plot

\begin{Shaded}
\begin{Highlighting}[]
\CommentTok{# Plot as a combination of jitter and violin plot}
\NormalTok{durationRatioPlot1 }\OperatorTok{+}\StringTok{ }\KeywordTok{geom_violin}\NormalTok{() }\OperatorTok{+}
\StringTok{    }\KeywordTok{geom_jitter}\NormalTok{(}\DataTypeTok{alpha =} \FloatTok{0.75}\NormalTok{)}
\end{Highlighting}
\end{Shaded}

\includegraphics{test_novel_object_analysis_files/figure-latex/unnamed-chunk-11-1.pdf}

\begin{Shaded}
\begin{Highlighting}[]
\CommentTok{# Save the plot}
\KeywordTok{ggsave}\NormalTok{(}\StringTok{"novelObject_durationObjectRatio_violin_jitter.png"}\NormalTok{)}
\end{Highlighting}
\end{Shaded}

Plot the Duration Object Ratio versus Trial Type as a beeswarm plot

\begin{Shaded}
\begin{Highlighting}[]
\CommentTok{# Plot as a combination of beeswarm and box plot}
\NormalTok{durationRatioPlot1 }\OperatorTok{+}\StringTok{ }\KeywordTok{geom_boxplot}\NormalTok{(}\DataTypeTok{outlier.color =} \StringTok{'purple'}\NormalTok{) }\OperatorTok{+}
\StringTok{    }\KeywordTok{geom_beeswarm}\NormalTok{(}\DataTypeTok{alpha =} \FloatTok{0.75}\NormalTok{)}
\end{Highlighting}
\end{Shaded}

\includegraphics{test_novel_object_analysis_files/figure-latex/unnamed-chunk-12-1.pdf}

\begin{Shaded}
\begin{Highlighting}[]
\CommentTok{# Save the plot}
\KeywordTok{ggsave}\NormalTok{(}\StringTok{"novelObject_durationObjectRatio_boxplot_beeswarm.png"}\NormalTok{)}
\end{Highlighting}
\end{Shaded}


\end{document}
