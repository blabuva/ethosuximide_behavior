\documentclass[]{article}
\usepackage{lmodern}
\usepackage{amssymb,amsmath}
\usepackage{ifxetex,ifluatex}
\usepackage{fixltx2e} % provides \textsubscript
\ifnum 0\ifxetex 1\fi\ifluatex 1\fi=0 % if pdftex
  \usepackage[T1]{fontenc}
  \usepackage[utf8]{inputenc}
\else % if luatex or xelatex
  \ifxetex
    \usepackage{mathspec}
  \else
    \usepackage{fontspec}
  \fi
  \defaultfontfeatures{Ligatures=TeX,Scale=MatchLowercase}
\fi
% use upquote if available, for straight quotes in verbatim environments
\IfFileExists{upquote.sty}{\usepackage{upquote}}{}
% use microtype if available
\IfFileExists{microtype.sty}{%
\usepackage{microtype}
\UseMicrotypeSet[protrusion]{basicmath} % disable protrusion for tt fonts
}{}
\usepackage[margin=1in]{geometry}
\usepackage{hyperref}
\hypersetup{unicode=true,
            pdftitle={Test Open Field Analysis},
            pdfauthor={Adam Lu},
            pdfborder={0 0 0},
            breaklinks=true}
\urlstyle{same}  % don't use monospace font for urls
\usepackage{color}
\usepackage{fancyvrb}
\newcommand{\VerbBar}{|}
\newcommand{\VERB}{\Verb[commandchars=\\\{\}]}
\DefineVerbatimEnvironment{Highlighting}{Verbatim}{commandchars=\\\{\}}
% Add ',fontsize=\small' for more characters per line
\usepackage{framed}
\definecolor{shadecolor}{RGB}{248,248,248}
\newenvironment{Shaded}{\begin{snugshade}}{\end{snugshade}}
\newcommand{\KeywordTok}[1]{\textcolor[rgb]{0.13,0.29,0.53}{\textbf{#1}}}
\newcommand{\DataTypeTok}[1]{\textcolor[rgb]{0.13,0.29,0.53}{#1}}
\newcommand{\DecValTok}[1]{\textcolor[rgb]{0.00,0.00,0.81}{#1}}
\newcommand{\BaseNTok}[1]{\textcolor[rgb]{0.00,0.00,0.81}{#1}}
\newcommand{\FloatTok}[1]{\textcolor[rgb]{0.00,0.00,0.81}{#1}}
\newcommand{\ConstantTok}[1]{\textcolor[rgb]{0.00,0.00,0.00}{#1}}
\newcommand{\CharTok}[1]{\textcolor[rgb]{0.31,0.60,0.02}{#1}}
\newcommand{\SpecialCharTok}[1]{\textcolor[rgb]{0.00,0.00,0.00}{#1}}
\newcommand{\StringTok}[1]{\textcolor[rgb]{0.31,0.60,0.02}{#1}}
\newcommand{\VerbatimStringTok}[1]{\textcolor[rgb]{0.31,0.60,0.02}{#1}}
\newcommand{\SpecialStringTok}[1]{\textcolor[rgb]{0.31,0.60,0.02}{#1}}
\newcommand{\ImportTok}[1]{#1}
\newcommand{\CommentTok}[1]{\textcolor[rgb]{0.56,0.35,0.01}{\textit{#1}}}
\newcommand{\DocumentationTok}[1]{\textcolor[rgb]{0.56,0.35,0.01}{\textbf{\textit{#1}}}}
\newcommand{\AnnotationTok}[1]{\textcolor[rgb]{0.56,0.35,0.01}{\textbf{\textit{#1}}}}
\newcommand{\CommentVarTok}[1]{\textcolor[rgb]{0.56,0.35,0.01}{\textbf{\textit{#1}}}}
\newcommand{\OtherTok}[1]{\textcolor[rgb]{0.56,0.35,0.01}{#1}}
\newcommand{\FunctionTok}[1]{\textcolor[rgb]{0.00,0.00,0.00}{#1}}
\newcommand{\VariableTok}[1]{\textcolor[rgb]{0.00,0.00,0.00}{#1}}
\newcommand{\ControlFlowTok}[1]{\textcolor[rgb]{0.13,0.29,0.53}{\textbf{#1}}}
\newcommand{\OperatorTok}[1]{\textcolor[rgb]{0.81,0.36,0.00}{\textbf{#1}}}
\newcommand{\BuiltInTok}[1]{#1}
\newcommand{\ExtensionTok}[1]{#1}
\newcommand{\PreprocessorTok}[1]{\textcolor[rgb]{0.56,0.35,0.01}{\textit{#1}}}
\newcommand{\AttributeTok}[1]{\textcolor[rgb]{0.77,0.63,0.00}{#1}}
\newcommand{\RegionMarkerTok}[1]{#1}
\newcommand{\InformationTok}[1]{\textcolor[rgb]{0.56,0.35,0.01}{\textbf{\textit{#1}}}}
\newcommand{\WarningTok}[1]{\textcolor[rgb]{0.56,0.35,0.01}{\textbf{\textit{#1}}}}
\newcommand{\AlertTok}[1]{\textcolor[rgb]{0.94,0.16,0.16}{#1}}
\newcommand{\ErrorTok}[1]{\textcolor[rgb]{0.64,0.00,0.00}{\textbf{#1}}}
\newcommand{\NormalTok}[1]{#1}
\usepackage{graphicx,grffile}
\makeatletter
\def\maxwidth{\ifdim\Gin@nat@width>\linewidth\linewidth\else\Gin@nat@width\fi}
\def\maxheight{\ifdim\Gin@nat@height>\textheight\textheight\else\Gin@nat@height\fi}
\makeatother
% Scale images if necessary, so that they will not overflow the page
% margins by default, and it is still possible to overwrite the defaults
% using explicit options in \includegraphics[width, height, ...]{}
\setkeys{Gin}{width=\maxwidth,height=\maxheight,keepaspectratio}
\IfFileExists{parskip.sty}{%
\usepackage{parskip}
}{% else
\setlength{\parindent}{0pt}
\setlength{\parskip}{6pt plus 2pt minus 1pt}
}
\setlength{\emergencystretch}{3em}  % prevent overfull lines
\providecommand{\tightlist}{%
  \setlength{\itemsep}{0pt}\setlength{\parskip}{0pt}}
\setcounter{secnumdepth}{0}
% Redefines (sub)paragraphs to behave more like sections
\ifx\paragraph\undefined\else
\let\oldparagraph\paragraph
\renewcommand{\paragraph}[1]{\oldparagraph{#1}\mbox{}}
\fi
\ifx\subparagraph\undefined\else
\let\oldsubparagraph\subparagraph
\renewcommand{\subparagraph}[1]{\oldsubparagraph{#1}\mbox{}}
\fi

%%% Use protect on footnotes to avoid problems with footnotes in titles
\let\rmarkdownfootnote\footnote%
\def\footnote{\protect\rmarkdownfootnote}

%%% Change title format to be more compact
\usepackage{titling}

% Create subtitle command for use in maketitle
\newcommand{\subtitle}[1]{
  \posttitle{
    \begin{center}\large#1\end{center}
    }
}

\setlength{\droptitle}{-2em}

  \title{Test Open Field Analysis}
    \pretitle{\vspace{\droptitle}\centering\huge}
  \posttitle{\par}
    \author{Adam Lu}
    \preauthor{\centering\large\emph}
  \postauthor{\par}
      \predate{\centering\large\emph}
  \postdate{\par}
    \date{December 7th, 2018}


\begin{document}
\maketitle

\subsection{Read in the raw data}\label{read-in-the-raw-data}

The file \textbf{Analysis Output-AV Test Open Field.csv} is TODO:

\begin{Shaded}
\begin{Highlighting}[]
\CommentTok{# Load libraries (tidyverse includes readr, dplyr, ggplot2 ... etc.)}
\KeywordTok{library}\NormalTok{(tidyverse)}
\KeywordTok{library}\NormalTok{(ggbeeswarm)}
\end{Highlighting}
\end{Shaded}

\begin{Shaded}
\begin{Highlighting}[]
\CommentTok{# Import open field data}
\NormalTok{openFieldDataRaw <-}\StringTok{ }\KeywordTok{read_csv}\NormalTok{(}\StringTok{"Analysis Output-AV Test Open Field.csv"}\NormalTok{, }\DataTypeTok{skip =} \DecValTok{4}\NormalTok{, }
                            \DataTypeTok{col_names =}\NormalTok{ header, }\DataTypeTok{na =} \KeywordTok{c}\NormalTok{(}\StringTok{""}\NormalTok{, }\StringTok{"NA"}\NormalTok{, }\StringTok{"-"}\NormalTok{))}

\CommentTok{# Take a look at the raw open field data}
\NormalTok{openFieldDataRaw}
\end{Highlighting}
\end{Shaded}

\begin{verbatim}
## # A tibble: 20 x 10
##    `Trial #` `In Border:Peri~ `In Border:Peri~ `In Border:Peri~
##        <dbl>            <dbl>            <dbl>            <dbl>
##  1         1             599.                1           1.27  
##  2         2             563.               15           0     
##  3         3             581.               14           0.267 
##  4         4             594.                7           0.534 
##  5         5             553.               13           0     
##  6         6             593.                4           0.334 
##  7         7             566.               24           0     
##  8         8             598.                4           0     
##  9         9             576.               13           0     
## 10        10             590.                4           0     
## 11        11             586.                6           5.47  
## 12        12             584.               13           0     
## 13        13             567.               25           0.0667
## 14        14             587.               13           0     
## 15        15             585.                9           0     
## 16        16             577.               13           0     
## 17        17             597.                4           0.334 
## 18        18             566.               23           1.40  
## 19        19             596.                5           0     
## 20        20             588.               11           0     
## # ... with 6 more variables: `In Center:Center:Duration` <dbl>, `In
## #   Center:Center:Frequency` <dbl>, `In Center:Center:Latency to
## #   first` <dbl>, `In Zone:Center/Center-point:Duration` <dbl>, `In
## #   Zone:Center/Center-point:Frequency` <dbl>, `In
## #   Zone:Center/Center-point:Latency to first` <dbl>
\end{verbatim}

\begin{Shaded}
\begin{Highlighting}[]
\CommentTok{# Import injections data}
\NormalTok{injectionsDataRaw <-}\StringTok{ }\KeywordTok{read_csv}\NormalTok{(}\StringTok{"Injections Data.csv"}\NormalTok{, }\DataTypeTok{n_max =} \DecValTok{20}\NormalTok{)}

\CommentTok{# Take a look at the injections data}
\NormalTok{injectionsDataRaw}
\end{Highlighting}
\end{Shaded}

\begin{verbatim}
## # A tibble: 20 x 29
##    Treatment Sex   Tag   `12-Oct` `13-Oct` `14-Oct` `15-Oct` `16-Oct`
##        <dbl> <chr> <chr>    <dbl>    <dbl>    <dbl>    <dbl>    <dbl>
##  1         0 F     1L          25       27       31       34       36
##  2         1 M     2L          27       29       33       37       36
##  3         0 M     3L          25       27       31       33       34
##  4         1 F     1R          23       24       28       30       33
##  5         0 M     2R          25       27       31       34       35
##  6         1 F     3R          21       23       25       27       30
##  7         0 M     1L-1R       27       30       34       36       37
##  8         1 M     1L-2R       28       29       32       36       38
##  9         0 F     2L-2R       26       27       31       33       35
## 10         1 F     2L-1R       27       28       31       34       36
## 11         0 F     1L          21       25       27       31       32
## 12         1 F     2L          23       25       28       30       33
## 13         0 M     3L          23       27       30       32       36
## 14         1 M     1R          24       26       28       32       34
## 15         0 M     2L-1R       23       26       28       31       33
## 16         1 M     3R          24       27       30       33       36
## 17         0 M     1L-1R       25       27       30       33       35
## 18         1 F     1L-2R       22       25       27       30       32
## 19         0 F     2L-2R       23       26       29       31       35
## 20         1 F     2R          23       26       29       30       33
## # ... with 21 more variables: `17-Oct` <dbl>, `18-Oct` <dbl>,
## #   `19-Oct` <dbl>, `20-Oct` <dbl>, `21-Oct` <dbl>, `22-Oct` <dbl>,
## #   `23-Oct` <dbl>, `24-Oct` <dbl>, `25-Oct` <dbl>, `26-Oct` <dbl>,
## #   `27-Oct` <dbl>, `28-Oct` <dbl>, `29-Oct` <dbl>, `30-Oct` <dbl>,
## #   `31-Oct` <dbl>, `Parent Cage #` <dbl>, `Current Cage #` <dbl>, `Trial
## #   # Open Field` <dbl>, `Trial # Familiar Object` <dbl>, `Trial # Novel
## #   Object` <dbl>, Comments <chr>
\end{verbatim}

Create a table that maps treatment numbers to treatment strings

\begin{Shaded}
\begin{Highlighting}[]
\CommentTok{# Create a dictionary for Treatment}
\NormalTok{treatmentTypeTable <-}\StringTok{ }
\StringTok{    }\KeywordTok{tibble}\NormalTok{(}\DataTypeTok{Treatment =} \DecValTok{0}\OperatorTok{:}\DecValTok{1}\NormalTok{, }\DataTypeTok{TreatmentStr =} \KeywordTok{c}\NormalTok{(}\StringTok{"Control"}\NormalTok{, }\StringTok{"Ethosuximide"}\NormalTok{))}

\CommentTok{# Display the table}
\NormalTok{treatmentTypeTable}
\end{Highlighting}
\end{Shaded}

\begin{verbatim}
## # A tibble: 2 x 2
##   Treatment TreatmentStr
##       <int> <chr>       
## 1         0 Control     
## 2         1 Ethosuximide
\end{verbatim}

\subsection{Combine and rearrage the
data}\label{combine-and-rearrage-the-data}

TODO

\begin{Shaded}
\begin{Highlighting}[]
\CommentTok{# Join the tables}
\NormalTok{openFieldDataExpanded <-}\StringTok{ }
\StringTok{    }\NormalTok{openFieldDataRaw }\OperatorTok\StringTok{ }
\StringTok{    }\KeywordTok{full_join}\NormalTok{(injectionsDataRaw, }\DataTypeTok{by =} \KeywordTok{c}\NormalTok{(}\StringTok{"Trial #"}\NormalTok{ =}\StringTok{ "Trial # Open Field"}\NormalTok{)) }\OperatorTok\StringTok{ }
\StringTok{    }\KeywordTok{full_join}\NormalTok{(treatmentTypeTable, }\DataTypeTok{by =} \StringTok{"Treatment"}\NormalTok{) }\OperatorTok\StringTok{ }
\StringTok{    }\KeywordTok{mutate}\NormalTok{(}\DataTypeTok{durationInCenter =} \StringTok{`}\DataTypeTok{In Zone:Center/Center-point:Duration}\StringTok{`}\NormalTok{) }\OperatorTok\StringTok{ }
\StringTok{    }\KeywordTok{mutate}\NormalTok{(}\DataTypeTok{durationInBorder =} \StringTok{`}\DataTypeTok{In Border:Periphery/Center-point:Duration}\StringTok{`}\NormalTok{) }\OperatorTok\StringTok{ }
\StringTok{    }\KeywordTok{mutate}\NormalTok{(}\DataTypeTok{Sex =} \KeywordTok{factor}\NormalTok{(Sex)) }\OperatorTok\StringTok{ }
\StringTok{    }\KeywordTok{mutate}\NormalTok{(}\DataTypeTok{TreatmentStr =} \KeywordTok{factor}\NormalTok{(TreatmentStr)) }\OperatorTok\StringTok{ }
\StringTok{    }\KeywordTok{mutate}\NormalTok{(}\DataTypeTok{durationRatio =}\NormalTok{ durationInBorder }\OperatorTok{/}\StringTok{ }\NormalTok{durationInCenter)}

\CommentTok{# Take a glimpse of all the variables in the data }
\KeywordTok{glimpse}\NormalTok{(openFieldDataExpanded)}
\end{Highlighting}
\end{Shaded}

\begin{verbatim}
## Observations: 20
## Variables: 42
## $ `Trial #`                                           <dbl> 1, 2, 3, 4...
## $ `In Border:Periphery/Center-point:Duration`         <dbl> 598.7988, ...
## $ `In Border:Periphery/Center-point:Frequency`        <dbl> 1, 15, 14,...
## $ `In Border:Periphery/Center-point:Latency to first` <dbl> 1.267934, ...
## $ `In Center:Center:Duration`                         <dbl> NA, 57.123...
## $ `In Center:Center:Frequency`                        <dbl> 0, 27, 24,...
## $ `In Center:Center:Latency to first`                 <dbl> NA, 88.755...
## $ `In Zone:Center/Center-point:Duration`              <dbl> NA, 37.103...
## $ `In Zone:Center/Center-point:Frequency`             <dbl> 0, 15, 13,...
## $ `In Zone:Center/Center-point:Latency to first`      <dbl> NA, 89.289...
## $ Treatment                                           <dbl> 0, 0, 1, 0...
## $ Sex                                                 <fct> F, F, F, M...
## $ Tag                                                 <chr> "2L-2R", "...
## $ `12-Oct`                                            <dbl> 26, 25, 21...
## $ `13-Oct`                                            <dbl> 27, 27, 23...
## $ `14-Oct`                                            <dbl> 31, 31, 25...
## $ `15-Oct`                                            <dbl> 33, 34, 27...
## $ `16-Oct`                                            <dbl> 35, 36, 30...
## $ `17-Oct`                                            <dbl> 38, 39, 32...
## $ `18-Oct`                                            <dbl> 41, 42, 34...
## $ `19-Oct`                                            <dbl> 43, 45, 36...
## $ `20-Oct`                                            <dbl> 47, 49, 40...
## $ `21-Oct`                                            <dbl> 52, 53, 44...
## $ `22-Oct`                                            <dbl> 58, 59, 50...
## $ `23-Oct`                                            <dbl> 63, 64, 56...
## $ `24-Oct`                                            <dbl> 69, 71, 62...
## $ `25-Oct`                                            <dbl> 76, 76, 67...
## $ `26-Oct`                                            <dbl> 83, 86, 78...
## $ `27-Oct`                                            <dbl> 87, 90, 81...
## $ `28-Oct`                                            <dbl> 93, 99, 90...
## $ `29-Oct`                                            <dbl> 99, 107, 9...
## $ `30-Oct`                                            <dbl> 106, 114, ...
## $ `31-Oct`                                            <dbl> 112, 123, ...
## $ `Parent Cage #`                                     <dbl> 1, 1, 1, 1...
## $ `Current Cage #`                                    <dbl> 1, 1, 1, 4...
## $ `Trial # Familiar Object`                           <dbl> 1, 2, 3, 4...
## $ `Trial # Novel Object`                              <dbl> 5, 6, 7, 8...
## $ Comments                                            <chr> NA, NA, NA...
## $ TreatmentStr                                        <fct> Control, C...
## $ durationInCenter                                    <dbl> NA, 37.103...
## $ durationInBorder                                    <dbl> 598.7988, ...
## $ durationRatio                                       <dbl> NA, 15.174...
\end{verbatim}

Remove the first row because TODO

\begin{Shaded}
\begin{Highlighting}[]
\CommentTok{# Select all but the first row}
\NormalTok{openFieldDataFiltered <-}\StringTok{ }
\StringTok{    }\NormalTok{openFieldDataExpanded }\OperatorTok\StringTok{ }
\StringTok{    }\KeywordTok{slice}\NormalTok{(}\DecValTok{2}\OperatorTok{:}\KeywordTok{n}\NormalTok{())}

\CommentTok{# Take a look at the data with rows filtered}
\NormalTok{openFieldDataFiltered}
\end{Highlighting}
\end{Shaded}

\begin{verbatim}
## # A tibble: 19 x 42
##    `Trial #` `In Border:Peri~ `In Border:Peri~ `In Border:Peri~
##        <dbl>            <dbl>            <dbl>            <dbl>
##  1         2             563.               15           0     
##  2         3             581.               14           0.267 
##  3         4             594.                7           0.534 
##  4         5             553.               13           0     
##  5         6             593.                4           0.334 
##  6         7             566.               24           0     
##  7         8             598.                4           0     
##  8         9             576.               13           0     
##  9        10             590.                4           0     
## 10        11             586.                6           5.47  
## 11        12             584.               13           0     
## 12        13             567.               25           0.0667
## 13        14             587.               13           0     
## 14        15             585.                9           0     
## 15        16             577.               13           0     
## 16        17             597.                4           0.334 
## 17        18             566.               23           1.40  
## 18        19             596.                5           0     
## 19        20             588.               11           0     
## # ... with 38 more variables: `In Center:Center:Duration` <dbl>, `In
## #   Center:Center:Frequency` <dbl>, `In Center:Center:Latency to
## #   first` <dbl>, `In Zone:Center/Center-point:Duration` <dbl>, `In
## #   Zone:Center/Center-point:Frequency` <dbl>, `In
## #   Zone:Center/Center-point:Latency to first` <dbl>, Treatment <dbl>,
## #   Sex <fct>, Tag <chr>, `12-Oct` <dbl>, `13-Oct` <dbl>, `14-Oct` <dbl>,
## #   `15-Oct` <dbl>, `16-Oct` <dbl>, `17-Oct` <dbl>, `18-Oct` <dbl>,
## #   `19-Oct` <dbl>, `20-Oct` <dbl>, `21-Oct` <dbl>, `22-Oct` <dbl>,
## #   `23-Oct` <dbl>, `24-Oct` <dbl>, `25-Oct` <dbl>, `26-Oct` <dbl>,
## #   `27-Oct` <dbl>, `28-Oct` <dbl>, `29-Oct` <dbl>, `30-Oct` <dbl>,
## #   `31-Oct` <dbl>, `Parent Cage #` <dbl>, `Current Cage #` <dbl>, `Trial
## #   # Familiar Object` <dbl>, `Trial # Novel Object` <dbl>,
## #   Comments <chr>, TreatmentStr <fct>, durationInCenter <dbl>,
## #   durationInBorder <dbl>, durationRatio <dbl>
\end{verbatim}

Select variables that will be analyzed

\begin{Shaded}
\begin{Highlighting}[]
\CommentTok{# Select variables that will be analyzed}
\NormalTok{openFieldData <-}\StringTok{ }
\StringTok{    }\NormalTok{openFieldDataFiltered }\OperatorTok\StringTok{ }
\StringTok{    }\KeywordTok{select}\NormalTok{(}\StringTok{`}\DataTypeTok{Trial #}\StringTok{`}\NormalTok{, TreatmentStr, Sex, durationInCenter, durationInBorder, durationRatio)}

\CommentTok{# Take a look at the data with columns filtered}
\NormalTok{openFieldData}
\end{Highlighting}
\end{Shaded}

\begin{verbatim}
## # A tibble: 19 x 6
##    `Trial #` TreatmentStr Sex   durationInCenter durationInBorder
##        <dbl> <fct>        <fct>            <dbl>            <dbl>
##  1         2 Control      F                37.1              563.
##  2         3 Ethosuximide F                18.4              581.
##  3         4 Control      M                 5.34             594.
##  4         5 Control      M                47.5              553.
##  5         6 Ethosuximide F                 7.21             593.
##  6         7 Ethosuximide F                34.0              566.
##  7         8 Ethosuximide M                 2.20             598.
##  8         9 Ethosuximide M                24.5              576.
##  9        10 Control      M                 9.81             590.
## 10        11 Ethosuximide F                 8.94             586.
## 11        12 Ethosuximide F                16.3              584.
## 12        13 Control      M                32.8              567.
## 13        14 Control      M                13.1              587.
## 14        15 Ethosuximide M                15.0              585.
## 15        16 Ethosuximide M                22.8              577.
## 16        17 Control      M                 2.80             597.
## 17        18 Control      F                32.6              566.
## 18        19 Ethosuximide F                 3.74             596.
## 19        20 Control      F                12.5              588.
## # ... with 1 more variable: durationRatio <dbl>
\end{verbatim}

\subsection{Make plots}\label{make-plots}

\paragraph{Use Duration in Center}\label{use-duration-in-center}

Plot the Duration in Center versus Treatment as a box plot

\begin{Shaded}
\begin{Highlighting}[]
\CommentTok{# Add the data to the canvas}
\NormalTok{durationPlot <-}\StringTok{ }\KeywordTok{ggplot}\NormalTok{(openFieldData, }\KeywordTok{aes}\NormalTok{(TreatmentStr, durationInCenter)) }\OperatorTok{+}
\StringTok{    }\KeywordTok{xlab}\NormalTok{(}\StringTok{"Treatment Group"}\NormalTok{) }\OperatorTok{+}\StringTok{ }\KeywordTok{ylab}\NormalTok{(}\StringTok{"Duration in Center (s)"}\NormalTok{) }\OperatorTok{+}
\StringTok{    }\KeywordTok{ggtitle}\NormalTok{(}\StringTok{"Open Field Test for Ethosuximide-treated animals"}\NormalTok{)}

\CommentTok{# Plot as a combination of jitter and box plot}
\NormalTok{durationPlot }\OperatorTok{+}\StringTok{ }\KeywordTok{geom_boxplot}\NormalTok{(}\DataTypeTok{outlier.color =} \StringTok{'red'}\NormalTok{, }\DataTypeTok{alpha =} \FloatTok{0.25}\NormalTok{) }\OperatorTok{+}
\StringTok{    }\KeywordTok{geom_jitter}\NormalTok{(}\DataTypeTok{alpha =} \FloatTok{0.25}\NormalTok{)}
\end{Highlighting}
\end{Shaded}

\includegraphics{test_open_field_analysis_files/figure-latex/unnamed-chunk-8-1.pdf}

\begin{Shaded}
\begin{Highlighting}[]
\CommentTok{# Save the plot}
\KeywordTok{ggsave}\NormalTok{(}\StringTok{"openField_durationCenter_boxplot_jitter.png"}\NormalTok{)}
\end{Highlighting}
\end{Shaded}

Plot the Duration in Center versus Treatment as a violin plot

\begin{Shaded}
\begin{Highlighting}[]
\CommentTok{# Plot as a combination of jitter and violin plot}
\NormalTok{durationPlot }\OperatorTok{+}\StringTok{ }\KeywordTok{geom_violin}\NormalTok{() }\OperatorTok{+}
\StringTok{    }\KeywordTok{geom_jitter}\NormalTok{(}\DataTypeTok{alpha =} \FloatTok{0.25}\NormalTok{)}
\end{Highlighting}
\end{Shaded}

\includegraphics{test_open_field_analysis_files/figure-latex/unnamed-chunk-9-1.pdf}

\begin{Shaded}
\begin{Highlighting}[]
\CommentTok{# Save the plot}
\KeywordTok{ggsave}\NormalTok{(}\StringTok{"openField_durationCenter_violin_jitter.png"}\NormalTok{)}
\end{Highlighting}
\end{Shaded}

Plot the Duration in Center versus Treatment as a beeswarm plot

\begin{Shaded}
\begin{Highlighting}[]
\CommentTok{# Plot as a combination of box and beeswarm plot}
\NormalTok{durationPlot }\OperatorTok{+}\StringTok{ }\KeywordTok{geom_boxplot}\NormalTok{(}\DataTypeTok{outlier.color =} \StringTok{'red'}\NormalTok{, }\DataTypeTok{alpha =} \FloatTok{0.25}\NormalTok{) }\OperatorTok{+}
\StringTok{    }\KeywordTok{geom_beeswarm}\NormalTok{(}\DataTypeTok{alpha =} \FloatTok{0.25}\NormalTok{)}
\end{Highlighting}
\end{Shaded}

\includegraphics{test_open_field_analysis_files/figure-latex/unnamed-chunk-10-1.pdf}

\begin{Shaded}
\begin{Highlighting}[]
\CommentTok{# Save the plot}
\KeywordTok{ggsave}\NormalTok{(}\StringTok{"openField_durationCenter_boxplot_beeswarm.png"}\NormalTok{)}
\end{Highlighting}
\end{Shaded}

\paragraph{Use Ratio of Duration in Border to Duration in
Center}\label{use-ratio-of-duration-in-border-to-duration-in-center}

\begin{Shaded}
\begin{Highlighting}[]
\CommentTok{# Add the data to the canvas}
\NormalTok{durationRatioPlot <-}\StringTok{ }\KeywordTok{ggplot}\NormalTok{(openFieldData, }\KeywordTok{aes}\NormalTok{(TreatmentStr, durationRatio)) }\OperatorTok{+}
\StringTok{    }\KeywordTok{xlab}\NormalTok{(}\StringTok{"Treatment Group"}\NormalTok{) }\OperatorTok{+}\StringTok{ }\KeywordTok{ylab}\NormalTok{(}\StringTok{"Duration in Border/Duration in Center"}\NormalTok{) }\OperatorTok{+}
\StringTok{    }\KeywordTok{ggtitle}\NormalTok{(}\StringTok{"Open Field Test for Ethosuximide-treated animals"}\NormalTok{)}

\CommentTok{# Plot as a combination of jitter and box plot}
\NormalTok{durationRatioPlot }\OperatorTok{+}\StringTok{ }\KeywordTok{geom_boxplot}\NormalTok{(}\DataTypeTok{outlier.color =} \StringTok{'red'}\NormalTok{, }\DataTypeTok{alpha =} \FloatTok{0.25}\NormalTok{) }\OperatorTok{+}
\StringTok{    }\KeywordTok{geom_jitter}\NormalTok{(}\DataTypeTok{alpha =} \FloatTok{0.25}\NormalTok{)}
\end{Highlighting}
\end{Shaded}

\includegraphics{test_open_field_analysis_files/figure-latex/unnamed-chunk-11-1.pdf}

\begin{Shaded}
\begin{Highlighting}[]
\CommentTok{# Save the plot}
\KeywordTok{ggsave}\NormalTok{(}\StringTok{"openField_durationRatio_boxplot_jitter.png"}\NormalTok{)}
\end{Highlighting}
\end{Shaded}

Plot the Ratio of Duration in Border to Duration in Center as a violin
plot

\begin{Shaded}
\begin{Highlighting}[]
\CommentTok{# Plot as a combination of jitter and violin plot}
\NormalTok{durationRatioPlot }\OperatorTok{+}\StringTok{ }\KeywordTok{geom_violin}\NormalTok{() }\OperatorTok{+}
\StringTok{    }\KeywordTok{geom_jitter}\NormalTok{(}\DataTypeTok{alpha =} \FloatTok{0.25}\NormalTok{) }\OperatorTok{+}

\CommentTok{# Save the plot}
\KeywordTok{ggsave}\NormalTok{(}\StringTok{"openField_durationRatio_violin_jitter.png"}\NormalTok{)}
\end{Highlighting}
\end{Shaded}

\includegraphics{test_open_field_analysis_files/figure-latex/unnamed-chunk-12-1.pdf}

Plot the Ratio of Duration in Border to Duration in Center as a beeswarm
plot

\begin{Shaded}
\begin{Highlighting}[]
\CommentTok{# Plot as a combination of beeswarm and box plot}
\NormalTok{durationRatioPlot }\OperatorTok{+}\StringTok{ }\KeywordTok{geom_boxplot}\NormalTok{(}\DataTypeTok{outlier.color =} \StringTok{'red'}\NormalTok{, }\DataTypeTok{alpha =} \FloatTok{0.25}\NormalTok{) }\OperatorTok{+}
\StringTok{    }\KeywordTok{geom_beeswarm}\NormalTok{(}\DataTypeTok{alpha =} \FloatTok{0.25}\NormalTok{)}
\end{Highlighting}
\end{Shaded}

\includegraphics{test_open_field_analysis_files/figure-latex/unnamed-chunk-13-1.pdf}

\begin{Shaded}
\begin{Highlighting}[]
\CommentTok{# Save the plot}
\KeywordTok{ggsave}\NormalTok{(}\StringTok{"openField_durationRatio_boxplot_beeswarm.png"}\NormalTok{)}
\end{Highlighting}
\end{Shaded}


\end{document}
